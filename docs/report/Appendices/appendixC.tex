\chapter{Code Examples}



\section{Distance between two coordinates}

Th Haversine Formula calculates the great circle distance between two given points on a sphere, the Earth, in this instance. It utilises spherical trigonometry, which allows it to calculate the distance between two points by using a spherical function and calculating the result from it, utilising a measure of the Earth's radius. For the purposes of the project, this is used to calculate the distance between the Dyfi Wildlife Centre and a second coordinate pair.

\begin{verbatim}
    /**
     * Calculates the distance between the given coordinates and the
     * coordinates of the Dyfi Wildlife Centre. This is calculated using the
     * Haversine formula.
     * <p>
     * Code adapted from
     * <a href="https://rosettacode.org/wiki/Haversine_formula#Java">here</a>.
     *
     * @return double containing the distance in miles to 4 significant
     * figures.
     */
    @Override
    public double calculateDistanceFromCentre() {
        final double EARTH_RADIUS_MILES = 3958.8; // Approximate radius of
        // Earth in miles, used to calculate the distance.

        /* Local variables used for cleaner code */
        double dyfiLat = 52.568774;
        double dyfiLng = -3.918031;

        double currentLat = this.getLatitude();
        double currentLng = this.getLongitude();

        if ((dyfiLat == currentLat) && (dyfiLng == currentLng)) {
            return 0; // If there is no difference between both coordinates,
            // distance of 0 is returned, to avoid unnecessary calculation.
        } else {
            double dLat = Math.toRadians(currentLat - dyfiLat);
            double dLng = Math.toRadians(currentLng - dyfiLng);
            dyfiLat = Math.toRadians(dyfiLat);
            currentLat = Math.toRadians(currentLat);

            double a =
                    Math.pow(Math.sin(dLat / 2), 2) + Math.pow(
                    Math.sin(dLng / 2), 2)
                    * Math.cos(dyfiLat)
                    * Math.cos(currentLat);
            double c = 2 * Math.asin(Math.sqrt(a));
            double result = (EARTH_RADIUS_MILES * c);
            /* Converts result into a BigDecimal that is then rounded to 4
            significant figures.
             */
            MathContext mathContext = new MathContext(4, RoundingMode.DOWN);
            BigDecimal bigDecimal = new BigDecimal(result, mathContext);
            return bigDecimal.doubleValue();
        }

    }
\end{verbatim}    

\section{Google Maps callback function}

This is a JavaScript function that performs clustering and adding of markers, it is loaded when the index page is loaded.

\begin{verbatim}
async function initMap() {
    const allPointsOfInterest = await getPointsOfInterest('/poi');
    const dyfiWildlifeCentre = {lat: 52.568774, lng: -3.918031};
     // Co-ordinates for the Cors Dyfi Nature Reserve,
    // which should be located at the centre of the map.
    map = new google.maps.Map(document.getElementById('map'), {
        zoom: 16,
        center: dyfiWildlifeCentre,
        mapTypeId: 'hybrid',
        disableDefaultUI: true,
        clickableIcons: false
    });
    /* Creates a marker clusterer. Initial markers are null as they are
     added iteratively
     */
    markerCluster = new MarkerClusterer(map,null,
    {imagePath: 'images/clusters/m'});
    allPointsOfInterest.forEach(
        poi => {
            const marker = new google.maps.Marker({
                position: {lat: poi.latitude, lng: poi.longitude},
                map: map,
                title: poi.name,
                description: poi.description,
                distanceFromCentre: poi.distanceFromCentre,
                category: poi.category
            });
            allMarkers.push(marker);
            markerCluster.addMarker(marker); // Iterative addition of a
            // marker to the cluster, that performs clustering automatically
            marker.addListener('click', function () {
                const element = document.getElementById('poiCard');
                element.querySelector('#poi_title')
                	.innerHTML = marker.title;
                element.querySelector('#poi_description')
                	.innerHTML = marker.description;
                element.querySelector('#poi_distance')
                	.innerHTML = marker.distanceFromCentre;
                const instance = M.Modal.init(element, {
                    dismissible: true,
                    inDuration: 500,
                    outDuration: 500
                });

                instance.open();
            });
        }
    );
}
\end{verbatim}

\section{POI card}

This is a card containing Point of Interest information, that is populated using the aforementioned function.

\begin{verbatim}
<div class="modal" id="poiCard" th:fragment="poiCard">
    <div class="modal-content">
        <div class="container-fluid">
            <div class="card large z-depth-0">
                <div class="card-image">
                    <img alt="An example image" class="lazyload"
                         data-src="images/osprey.jpeg">
                    <span class="card-title black-text" id="poi_title"></span>
                </div>
                <div class="card-content grey lighten-4">
                    <p>This place is
                        <span id="poi_distance"
                              style="font-weight:bold"></span>
                        miles from the
                        Dyfi
                        Wildlife Centre</p>
                </div>
                <div class="card-content" id="scrollable">
                    <p id="poi_description"></p>
                </div>
            </div>
        </div>
    </div>
</div>
\end{verbatim}
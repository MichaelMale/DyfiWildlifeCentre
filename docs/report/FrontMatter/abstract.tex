\thispagestyle{empty}

%TC:ignore

\section*{\centering Abstract}

The Dyfi Wildlife Centre is a visitor centre run by the Montgomeryshire Wildlife Trust. It is situated on the Cors Dyfi Nature Reserve in Powys, Wales. The Trust had approached the University, with a request for a solution to assist in showcasing the reserve's work, and its place as an osprey conservation, engagement, and research project. They had procured an 86-inch touchscreen monitor, in the hopes of presenting an application which aids in achieving the aforementioned goal.

The software created in this project takes the form of a map-based web application. The front-end makes use of the Google Maps for JavaScript API, as well as HTML, CSS and Thymeleaf, to show visitors a map of the centre and its surroundings. The map has various markers, and filters, that can be clicked on, showing further information about the point of interest. This is backed up by a RESTful API backend, created using the Spring Framework, a model-view-controller framework using Java Enterprise Edition. An administration panel was also developed, involving authentication, and allowing the authenticated user to add, edit, and delete points of interest. Data was stored in a PostgreSQL relational database.

Planning and development of this project took the form of an agile approach, with ideas from both Kanban and Extreme Programming used and adapted to fit a single-developer project. A meeting with the customer took place prior to development, and user stories and prioritisation of tasks had branched out from that meeting. Elements of test-driven development were utilised in this project, with a mixture of manual testing tables and automated testing suites being used. A key part of this project was ensuring that the customer was kept up-to-date, and supplementary documentation was created for this project that provided a brief overview of the required setup and how to use the web app.



%TC:endignore
\documentclass[11pt,fleqn,twoside]{article}
\usepackage{makeidx}
\makeindex
\usepackage{palatino} %or {times} etc
\usepackage{plain} %bibliography style
\usepackage{amsmath} %math fonts - just in case
\usepackage{amsfonts} %math fonts
\usepackage{amssymb} %math fonts
\usepackage{lastpage} %for footer page numbers
\usepackage{fancyhdr} %header and footer package
\usepackage{mmpv2}
%\usepackage{url}
\usepackage{hyperref}

% the following packages are used for citations - You only need to include one.
%
% Use the cite package if you are using the numeric style (e.g. IEEEannot).
% Use the natbib package if you are using the author-date style (e.g. authordate2annot).
% Only use one of these and comment out the other one.
\usepackage{cite}
%\usepackage{natbib}

\begin{document}

\name{Michael Male}
\userid{mim39}
\projecttitle{Development of a map-based web application to be used by visitors and staff at the Dyfi Wildlife Centre}
\projecttitlememoir{Dyfi Wildlife Centre Web App} %same as the project title or abridged version for page header
\reporttitle{Project Outline}
\version{1.0}
\docstatus{Draft} % change to Release when you are ready to submit your document
\modulecode{CS39440}
\degreeschemecode{G40F}
\degreeschemename{Computer Science}
\supervisor{Edel Sherratt} % e.g. Neil Taylor
\supervisorid{eds} % e.g. nst

%optional - comment out next line to use current date for the document
%\documentdate{8th February 2019}
\mmp

%\setcounter{tocdepth}{3} %set required number of level in table of contents


%==============================================================================
\section{Project description}
%==============================================================================

The Dyfi Wildlife Centre is a visitor centre run by the Montgomeryshire Wildlife Trust, situated on the Cors Dyfi Nature Reserve in Powys, Wales. It's purpose is to showcase the reserve's work, and its place as an osprey conservation, engagement, and research project\cite{DyfiWildlifeCentre}. This project will aim to create a map-based application to assist volunteers at the centre while they aim to provide an interactive and educational experience to visitors. The application should operate from a browser on a touch-screen Windows PC, with information and details about the site presented in a user-friendly manner.

The project will develop a single-page web application to assist volunteers in providing information to visitors. Volunteers have varying levels of computer literacy, and importance must be placed on the application being easy to use and aesthetically pleasing. The interface will involve the use of a map API, such as the OpenStreetMap API\cite{OSM} or Google Cloud Maps Platform\cite{GoogleMapsPlatform}. An interface for adding information about specific points of interest will be developed; be that parts of the centre, public transport links, or local businesses surrounding the nature reserve. The application should also provide access to the Dyfi Osprey Project's existing webcam infrastructure, built to show a live feed of their osprey nests.

The project will also develop an administration portal, where volunteers can enter and manage information about the nature reserve through a graphical interface. Initial information, and the requirements of this administration portal, will have to be sought from the customer at some stage during the project. Information will be stored via persistent data.

The application will be used directly by the Dyfi Wildlife Centre. There is a need for the application to be maintainable and written with a view to long-term use. The project will place focus on sustainable software, along with the development of documentation for both end users and developers on how to use and maintain the application. Future use-cases for the project may involve it being used independently by visitors, or it being used on other devices such as mobile phones, and the project will consider how to allow for future iterations upon it to be as simple as possible.

The project will utilise an adapted form of Agile Development for one-person software development \cite{nystrom_2011}, which will be further adapted for an academic software project. The project will place emphasis on Test-Driven Development, with the construction of a CI/CD pipeline to assist in delivering working software.

%==============================================================================
\section{Proposed tasks}
%==============================================================================


%==============================================================================
\section{Project deliverables}
%==============================================================================

% %
% % Start to comment out / remove the following lines. They are only provided for instruction for this example template.  You don't need the following section title, because it will be added as part of the bibliography section.
% %
% %==============================================================================
% \section*{Your Bibliography - REMOVE this title and text for final version}
% %==============================================================================
% %
% You need to include an annotated bibliography. This should list all relevant web pages, books, journals etc. that you have consulted in researching your project. Each reference should include an annotation.

% The purpose of the section is to understand what sources you are looking at.  A correctly formatted list of items and annotations is sufficient. You might go further and make use of bibliographic tools, e.g. BibTeX in a LaTeX document, could be used to provide citations, for example \cite{NumericalRecipes} \cite{MarksPaper} \cite[99-101]{FailBlog} \cite{kittenpic_ref}.  The bibliographic tools are not a requirement, but you are welcome to use them.

% You can remove the above {\em Your Bibliography} section heading because it will be added in by the renewcommand which is part of the bibliography. The correct annotated bibliography information is provided below.
% %
% % End of comment out / remove the lines. They are only provided for instruction for this example template.
% %


\nocite{*} % include everything from the bibliography, irrespective of whether it has been referenced.

% the following line is included so that the bibliography is also shown in the table of contents. There is the possibility that this is added to the previous page for the bibliography. To address this, a newline is added so that it appears on the first page for the bibliography.
\newpage
\addcontentsline{toc}{section}{Initial Annotated Bibliography}

%
% example of including an annotated bibliography. The current style is an author date one. If you want to change, comment out the line and uncomment the subsequent line. You should also modify the packages included at the top (see the notes earlier in the file) and then trash your aux files and re-run.
%\bibliographystyle{authordate2annot}
\bibliographystyle{IEEEannotU}
\renewcommand{\refname}{Bibliography}  % if you put text into the final {} on this line, you will get an extra title, e.g. References. This isn't necessary for the outline project specification.
\bibliography{mmp} % References file

\end{document}
